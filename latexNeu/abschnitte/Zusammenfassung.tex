\section{Zusammenfassung}
%todo Kapitel 6 ist jetzt sehr kurz - bitte noch tiefer vergleichen mit related work, und schauen das du noch besser verbesserungen ilustrierst, zb. im Bereich view management, wo das ganze weiter genutzt werden kann (Leute mit seh-hilfen), etc.

In diesem Kapitel werden die bearbeiteten Themen kurz zusammengefasst.

\subsection{Konzepte und implementierte Funktionen}

In dieser Arbeit eine automatische Objekterkennung und Markierung für AR-Umgebungen mithilfe von Image Based Object Detection umgesetzt. 

Die Umgebung der AR-Brille wird mit RGB-Fotos aufgenommen. Diese Fotos werden mit Machine Learning Modellen nach Objekten durchsucht. Die Positionen der gefundene Objekte werden in der AR-Umgebung lokalisiert und mit einem Label markiert. Objekte werden durch nur maximal ein Label annotiert.  

Das automatische Erkennen und Labeln von Objekten kann für große und dynamische Umgebungen verwendet werden. Es bietet somit eine Grundlage für AR Anwendungen die in einer solchen Umgebung arbeiten sollen oder eine ausgeprägtes semantisches Verständnis der Umgebung benötigen.

 %Beispielsweise eine Blindenführung in unbekannten Umgebungen oder Applikationen aus den Gebieten Autonomous Driving und Robotics. \citep{cars} 



\subsection{Vergleich mit Related Works}
%todo this diesen teil detailierter
\subsubsection*{Objekterkennung und Labeling in Echtzeit}
Anders als das Framework von \cite{LabelingLanguageLearning} läuft die vorgestellte Objekterkennung nicht in Echtzeit. \cite{LabelingLanguageLearning} verwendet eine niedrige Kameraauflösung und Bildqualität, damit die Bildanalyse in 30 ms durchgeführt werden kann.

Unser Ansatz zielt nicht auf Echtzeit ab, sondern auf ein breiteres semantisches Verständnis. Daher werden, anders als bei \cite{LabelingLanguageLearning}, mehrere neuronale Netzwerke verwendet. In unserer Applikation wird Azure Objekt Detection für den Großteil der Objekterkennung verwendet. Dieses Netz wird durch Azure Custom Prediction unterstützt. Durch die Kombination können insgesamt mehr Objekte erkannt werden. Der Nachteil in diesem Vorgenen besteht darin, das jedes der neuronalen Netze Zeit braucht um eine Analyse durchzuführen, unabhängig davon wie viele Objekte sie finden. Die Objekterkennung dauert damit insgesamt länger.

In unserem Verfahren wird die höchste Kameraauflösung für die RGB-Bilder verwendet. Die Bilder beinhalten mehr Informationen, um den neuronalen Netzwerken die beste Möglichkeit zu geben Objekte zu erkennen. Die größeren Bilddateien brauchen mehr Zeit um sie an den Server zu verschicken und analysiert zu werden.\citep{LabelingLanguageLearning}

\subsubsection*{Image Segmentation für kontextbezogene Interaktionen}
Das Framework von \cite{contextawaremixedreality} nutzt Image Segmentation, um jedem Voxel der Umgebung eine semantische Bedeutung zuzuweisen. Die Segmentierung findet, so wie Objekterkennung, semantische Informationen einer AR Umgebung. Die Informationen unterscheiden sich jedoch darin, auf welcher Ebene sie angesetzt sind. Die Informationen, die \cite{contextawaremixedreality} erheben, sind low-level im Vergleich mit den Informationen durch Objekterkennung. In dem Beispiel der Shooter-Anwendung wird den Voxeln jeweils ein Material zugewiesen. Diese semantische Information ist vergleichsweise low-level und lässt sich sehr gut mit dem Szeneverständnis durch Objekterkennung kombinieren. So kann beispielsweise für jedes Objekt angegeben werden aus welchem Material es besteht.\citep{contextawaremixedreality}



\subsubsection*{View Management}
Das View Management in unserer Applikation ist sehr minimal. Die Labels sind als 3D Objekte in der Szene umgesetzt. Sie rotieren, damit ihre Schriftzug immer dem Nutzer zugewandt ist. Die Position der Schriftzüge, relativ zu dem Objekt das sie annotieren, wird nicht verändert. Es wird nicht sichergestellt, das die Labels keine relevanten virtuellen oder realen Informationen verdecken. 

Die Labels werden nur mit einer neuen Position aktualisiert, wenn ihr Anker versetzt wird.

\subsection{Ausblick}
Die automatische Erkennung von Objekten in einer AR Umgebung ist funktionsfähig und effektiv.

Um die Lokalisierung eines Objektes in der AR Umgebung zu verbessern, können die geometrischen Informationen einer Spatial Map durch rohe Daten einer Tiefenkamera ergänzt werden. Da es rechenintensiv ist, die Spatial Map zu erstellen, kann es dazu kommen, dass stellenweise die Map noch nicht aufgebaut ist oder in einem veralteten Zustand vorliegt, wenn ein Objekt lokalisiert werden soll. Die rohen Daten der Tiefenkamera könnten dazu verwendet werden, die Spatial Map zu ergänzen.\citep{depththoughimage}

Zusätzlich kann die vorgestellte Applikation durch das Einsetzten von View Management verbessert werden. Dadurch kann sichergestellt werden, dass die gesetzten Labels immer lesbar sind. Durch die Objekterkennung werden Bereiche der realen Welt ausgewiesen, die relevant sind, da sie semantische Informationen enthalten. Mithilfe von View Management Methoden kann sichergestellt werden, dass diese relevanten Bereich nicht verdeckt werden.

Je nachdem welche Anwendung mit der Objekterkennung unterstützt werden soll, ist es wichtig, das die Informationen für einen Menschen lesbar sind. Die Applikationen von \cite{LabelingLanguageLearning} beruht beispielsweise darauf, Objekte gut sichtbar mit Labels zu versehen, um das Lernen einer Fremdsprache zu unterstützen.

Die Objekte, die von der Applikation erkannt werden, hängen von den verwendeten Machine Learning Modellen ab. In Zukunft könnte man weitere Modelle verwenden, um die Fotos der Umgebung zu analysieren. Durch Image Segmentation und Image Klassifikation können die semantischen Informationen einer Szene erweitert werden. Beispielsweise können Kontextinformationen verwendet werden, aus denen hervorgeht, in welchem Raumes eines Gebäudes die AR Brille eingesetzt wird (Küche, Arbeitszimmer oder Schlafzimmer). Und durch Image Segmentation kann beispielsweise festgestellt werden aus welchen Materialien die Objekte der Umgebung bestehen. Insgesamt kann das semantische Verständnis der Szene erweitert werden durch das einsetzten von mehreren unterschiedlichen Bildanalysen.

Durch das automatische erkennen von Objekten in einer Umgebung, wird es möglich AR Anwendungen zu realisieren, die darauf beruhen die Objekte in dynamischen und großen Umgebung zu kennen. Beispielsweise kann eine Blindenführung entwickelt werden. Das semantische Verständnis der Umgebung kann genutzt werden, um einen Menschen durch einen Raum zu führen. Sowohl geometrische als auch semantische Informationen werden dabei verwendet, um eine Umgebung zu beschreiben und Anweisungen des Nutzers zu interpretieren.

Eine andere Anwendungsmöglichkeit wäre ein künstliches fotografisches Gedächtnis. Wenn semantische Informationen und insbesondere Objekte der Umgebung über einen längeren Zeitraum gespeichert werden, kann eine AR Anwendung wiedergeben, wann ein bestimmtes Objekt zuletzt gesehen wurde und in welchem Kontext es sich befunden hat. Erweitert können periodisch Fotos der Umgebung aufgenommen werden und angezeigt werden, wenn der Nutzer nach Objekten fragt, die dort abgebildet sind.
