\section{Zusammenfassung}
%sachen die verbessert werden könnten
%todo this part.

In diesem Kapitel werden die bearbeiteten Themen kurz zusammengefasst.

\subsection{Konzepte und Implementierte Funktionen}

In dieser Arbeit wird Image Based Objekt Detection in eine Augmented Reality Brille integriert, um eine automatische Objekterkennung und Markierung von Objekten zu ermöglichen.

Es werden RGB-Fotos von der Umgebung der AR Brille aufgenommen und mit Machine Learning Modellen analysiert.
Das Modelle ist darauf trainiert Objekte und Lebewesen in einem Bild zu erkennen. Durch die Analyse werden Objekte auf den RGB-Fotos erkannt. Diese werden dann in der AR Umgebung lokalisiert und mit einem Label markiert.

Das automatische Erkennen und Labeln von Objekten kann für große und dynamische Umgebungen verwendet werden. Es bietet somit eine Grundlage für AR Anwendungen die in einer solchen Umgebung arbeiten sollen oder eine ausgeprägtes semantisches Verständnis der Umgebung benötigen. Beispielsweise eine Blindenführung in unbekannten Umgebungen oder Applikationen aus den Gebieten Autonomous Driving und Robotics. 

\subsection{Ausblick}

Die automatische Erkennung von Objekten in einer AR Umgebung ist funktionsfähig und effektiv.

Die Objekte, die erkannt werden, hängen von dem verwendeten Machine Learning Modell ab. In Zukunft könnte man weitere Modelle verwenden, um die Fotos der Umgebung zu analysieren. Durch Image Segmentation und Image Klassifikation können weitere Informationen erhoben werden. Durch das Inkludieren von Kontextinformationen kann beispielsweise hervorgehen, in welchem Raum eines Hauses sich die AR Brille befindet (Küche, Arbeitszimmer, Schlafzimmer).

Um die Lokalisierung eines Objektes in der AR Umgebung zu verbessern, können die geometrischen Informationen einer Spatial Map durch rohe Daten einer Tiefenkamera ergänzt werden. Da es rechenintensiv ist, die Spatial Map zu erstellen, kann es dazu kommen, dass stellenweise die Map noch nicht aufgebaut ist oder in einem veralteten Zustand vorliegt, wenn ein Objekte lokalisiert werden soll. Die rohen Daten der Tiefenkamera könnten dazu verwendet werden, die Spatial Map zu ergänzen. 
