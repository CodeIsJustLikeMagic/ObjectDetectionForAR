\section{Zusammenfassung}
%sachen die verbessert werden könnten
%todo this part.

In diesem Kapitel werden die bearbeiteten Themen kurz zusammengefasst.

\subsection{Konzepte und Impelemntierte Funktionen}

In dieser Arbeit wird Image based Objekt Detection in eine Augmented Reality Brille integriert, um eine automatisches Objekt Erkennen und Labeln zu ermöglichen.

Es werden Fotos von der Umgebung der AR Brille aufgenommen und mit einer Machine Learning Modell analysiert.
Das Modell ist darauf trainiert Objekte und Lebewesen in einem Bild zu erkennen. Objekte auf den Fotos der Umgebung werden so auf den Fotos erkannt. Dann werden sie in der AR Umgebung lokalisiert und mit einem Label Markiert.

Das automatische der Erkennen und Labeln von Objekten kann für große und dynamische Umgebungen verwendet werden. Es bietet somit eine gute Grundlage für AR Anwendungen die einer solchen Umgebung arbeiten oder eine ausgeprägteres Semantisches Verständnis der Umgebung benötigen. Beispielsweise eine Blindenführung in unbekannten Umgebungen, der Bereich des Autonemous Driving und Robotic. 

\subsection{Ausblick}

Die Automatische Erkennung von Objekten in einer AR Umgebung ist funktionsfähig und effektiv.

Die Objekte, die erkannt werden hängen von dem verwendeten Machine Learning Modell ab. Es gibt die Arten der Objekte und die Genauigkeit der Detection an. %Wie kann man das verbessern?

In Zukunft könnte man weitere Modelle verwenden um die Fotos der Umgebung zu analysieren. Beispielsweise können Kontextinformationen extrahiert werden. Daraus kann hervorheben, in welchem Raum eines Hauses die AR Brille sich befindet (Küche, Arbeitszimmer, Schlafzimmer).

Um die Lokalisierung eines Objektes in der AR Umgebung zu verbessern, könnte zusätzlich zu den Daten des Spatial Mappings, die Tiefenkamera der AR Brille verwendet werden. Die Spatial Map zu erstellen ist Arbeitsaufwändig. Das kann dazu führen Stellenweise die Map noch nicht aufgebaut ist, oder in einem Veralteten zustand vorliegt, wenn ein Objekte lokalisiert werden soll. Die rohen Daten der Tiefenkamera könnten dazu verwendet werden, das Spatial Map zu ergänzen und zu korrigieren.
